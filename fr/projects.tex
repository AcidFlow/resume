\cvline{}{
	\textbf{Application de visualisation de données d'un procès verbal de jury}
	\newline
	\newline
	Ce projet s'inscrit dans le cadre de la deuxième année d'étude à l'ENSEIRB-MATMECA. Il s'étend sur une année par équipe de 7 personnes, mise en relation avec un client. Il couvre toute l'étendu d'un projet, de la création du cahier des charges à la remise du produit fini au client.
	\newline
	L'application devait permettre de visualiser des informations extraites d'un procès verbal de jury afin d'améliorer la lisibilité et le traitement d'un tel document.
	\newline
	\textit{Technologies utilisées : Java, Java Swing, Hibernate, SQLite.}
}

\cvline{}{
	\textbf{Application d’échange de Fichiers en Pair à Pair}
	\newline
	\newline
	D'une durée de 2 mois, ce projet de deuxième année s'inscrit dans le cadre de l'apprentissage des protocoles réseaux. Réalisé en équipe de 4 personnes, il consistait à développer un système d'échange de fichiers centralisé puis décentralisé. 
	\newline
	\textit{Technologies utilisées : Java, C.}	
}

\cvline{}{
	\textbf{Bibliothèque de thread}
	\newline
	\newline
	Ce projet de système d'exploitation consistait à développer par équipe de 5 personnes une bibliothèque de thread en espace utilisateur.
	\newline
	\textit{Technologie utilisée : C.}	
}

